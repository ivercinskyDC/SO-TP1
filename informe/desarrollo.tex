\section{Desarrollo}

\subsection{ListaAtomica}

La $ListaAtomica$ es una implementación de lista usando variables atómicas. En particular, expone el método $push\_front$que se encarga de lidiar con llamadas concurrentes, de esta forma, actualiza la cabeza de la lista que es atómica utilizando $compare\_exchange\_strong$.

A su vez, sus elementos se leen usando la función $load$ para así obtener el dato correcto.

\subsection{ConcurrentHashMap}

La estructura $ConcurrentHashMap$ cuenta con un arreglo de listas atómicas representando cada una de las letras del alfabeto. Además cuenta con un arreglo de mutex con un mutex por letra. 

se implementaron los siguientes constructores y funciones:\\

\subsubsection{ConcurrentHashMap()}
\begin{codesnippet}
ConcurrentHashMap()
 	-inicializar 26 listas atómicas.
 	-inicializar los 26 mutex.
\end{codesnippet}\\

\subsubsection{ConcurrentHashMap(std::string arch)}
Crea el ConcurrentHashMap y carga en su estructura el archivo pasado de forma no concurrente.\\
\begin{codesnippet}
ConcurrentHashMap(string arch)
 	-Inicializar 26 listas atómicas.
 	-Inicializar los 26 mutex.
 	-process_file(arch)
\end{codesnippet}\\

\subsubsection{ConcurrentHashMap(unsigned int nt, std::list<std::string> archs)}
Crea un ConcurrentHashMap de forma concurrente corriendo $nt$ threads en paralelo que procesan los archivos.\\
\begin{codesnippet}
 ConcurrentHashMap(int nt, list<string> archs)
 	-Inicializar 26 listas atómicas.
 	-Inicializar los 26 mutex.
 	-declarar atomico para el indice de las listas
 	-lanzar nt threads
 	-esperar la finalización de los threads
\end{codesnippet}\\
Para esto se utiliza una variable atómica inicializada en -1 que va a representar el indice en la lista de archivos.
Cada uno de los threads, pide un índice y mientras luego procesa el íesimo archivo\\
\begin{codesnippet}
 	Thread:
 		i = indice.get_and_inc
 		mientras i sea menor a la cantidad de archivos
 			process_file(archivos[i])
 			i = indice.get_and_inc
\end{codesnippet}\\

\subsubsection{process_file(std::string arch)}
Procesa las palabras de cada archivo y las agrega al ConcurrentHashMap llamando a la función $add\_and\_inc$.

\subsubsection{add_and_inc(std::string key, int amount)}
Esta función se fija si la palabra procesada ya se encuentra en la estructura y aumenta su valor. En caso de no estar creada, la crea. Antes de comenzar lockea el mutex de la lista.\\
\begin{codesnippet}
 	-lockea el mutex correspondiente a la letra de la palabra.
 	-recorre la lista correspondiente para ver si existe la palabra.
 	-si existe, incrementa su valor en amount.
 	-si no existe la palabra, crea el par para la palabra y lo agrega a la lista.
 	-libera el mutex correspondiente a la letra de la palabra.
\end{codesnippet}\\

\subsubsection{ConcurrentHashMap::member(std::string key)}
Devuelve un booleano indicando si la palabra indicada está presente en la estructura, recorriendo la lista correspondiente a la palabra.

\subsubsection{ConcurrentHashMap::maximum(unsigned int nt)}
Calcula cual es la palabra que aparece con más repeticiones en la estructura, corriendo concurrentemente en $nt$ threads.\\
\begin{codesnippet}
 	-declara una ListaAtomica de pares y un atomico que va a ser el indice en los threads
 	-lockea los mutex de todas las letras
 	(para que nadie pueda agregar palabras mientras se calcula un máximo)
 	-lanza nt threads, cada uno agregara sus máximos a la lista atómica.
 	-espera la finalización de los nt threads.
 	-recorre la lista de máximos y devuelve el mayor
\end{codesnippet}\\
Cada uno de los threads funciona de la siguiente manera:\\
\begin{codesnippet}
 	Thread:
 		i = indice.get_and_inc
 		mientras i sea menor a la cantidad de letras
 			bucar maximo en la lista de la letra correspondiente
 			agregar maximo a la lista atómica de maximos
 			i = indice.get_and_inc

\end{codesnippet}\\
% subsubsection concurrenthashmap_maximum_unsigned_int_nt_ (end)
\subsubsection{ConcurrentHashMap::maximum(int p\_archivos,int p_maximos,
 list<string> archs)}
Dada una cantidad de archivos que se procesan en $p\_archivos$ threads, la función devuelve el máximo calculado en $p\_maximos$ threads.
Esto es realizado creando $p\_archivos$ ConcurrentHashMaps y lanzando un thread por cada hashmap creado.
Mediante un atómico que indica el indice dentro de la lista de archivos, los distintos hashmaps van procesando en su estructura los archivos.
Una vez terminado, se mergean todos los ConcurrentHashMaps en un solo ConcurrentHashMap. Esto se logra lanzando una cantidad de threads (en nuestra implementación una por letra del abecedario), donde cada uno de los threads procesa la lista correspondiente a una de las letras en todos los hashmaps, indicada mediante un indice atómico compartido por todos los threads.
Cuando se obtiene el ConcurrentHashMap con todos los archivos, se llama a la función maximum que devuelve el máximo.\\
\begin{codesnippet}
 	-declara en vector con p_archivos hashmaps
 	-lanza los thread en los que se procesan los archivos
 	-espera la finalización de los threads.
 	-lanza los threads que mergean los p_archivos hashmaps en uno solo.
 	-espera la finalización de los threads.
 	-llama a la función maximum del ConcurrentHashMap resultante
\end{codesnippet}\\
Cada uno de los threads funciona de la siguiente manera:\\
\begin{codesnippet}
 	ThreadProcesarHashmap:
 		i = indice.get_and_inc
 		mientras i sea menor a la cantidad de archivos
 			hm->process_file(archivos(i))
 			i = indice.get_and_inc

\end{codesnippet}\\
\\
\begin{codesnippet}
 	ThreadMergearHashmaps:
 		i = indice.get_and_inc
 		mientras i sea menor a la cantidad de letras
 			por cada hashMap en HashMaps
 				por cada elemento de la lista i
 					hm.add_and_inc(elem.clave,elem.valor)
 			i = indice.get_and_inc
\end{codesnippet}\\

\subsubsection{ConcurrentHashMap::concurrent_maximum(int p_archivos, int p_maximos,
list<string> archs)}
Calcula el maximo de forma concurrente, crea un nuevo ConcurrentHashMap utilizando $p_archivos$ threads para leer los archivos y le calcula el máximo utilizando $p_maximos$ threads.

